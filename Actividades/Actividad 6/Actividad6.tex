\documentclass{article}
\usepackage{amsmath}

\begin{document}
Para usar el método de cofactores (o expansión de Laplace), es necesario seleccionar una fila o columna cualquiera de una matriz cuadrada $A$ de orden $n$ para la cual se aplicará la siguiente fórmula

\begin{equation}
    det(A_{nxn})
    =
    \begin{vmatrix}
        a_{11} & a_{12} & \cdots & a_{1n} \\
        a_{21} & a_{22} & \cdots & a_{2n} \\
        \vdots & \vdots & \ddots & \vdots \\
        a_{n1} & a_{n2} & \cdots & a_{nn} \\
    \end{vmatrix}
    =
    \sum_{j=1}^{n}(-1)^{i+j}a_{i,j}m_{i,j}
\end{equation}

donde $a_{i,j}$ es el elemento en la fila $i$ y columna $j$ de la matriz $A$ y $m_{i,j}$ es el determinante de la menor obtenida al eliminar la fila $i$ y columna $j$ de $A$. Este es el caso al seleccionar una fila determinada $i$, pero si en su lugar se quisiera usar una columna $j$, basta con hacer la sumatoria desde $i=1$ hasta $n$.

\vspace{5mm}

Para utilizar la regla de Sarrus (en el caso de una matriz de orden 3) es requerido copiar las primeras dos columnas de la matriz (de izquierda a derecha) en el lado derecho de la misma. Posteriormente se sumará el producto de los elementos de cada diagonal descendente de "longitud tres" y se restará dicho producto pero de las diagonales ascendentes, de esta manera:

\begin{equation}
    \begin{aligned}
        \left(
        \begin{matrix}
            a_{11} & a_{12} & a_{13} \\
            a_{21} & a_{22} & a_{23} \\
            a_{31} & a_{32} & a_{33} \\
        \end{matrix}
        \middle|
        \begin{matrix}
            a_{11} & a_{12} \\
            a_{21} & a_{22} \\
            a_{31} & a_{32} \\
        \end{matrix}
        \right)
        &= a_{11}a_{22}a_{33} + a_{12}a_{23}a_{31} + a_{13}a_{21}a_{32} \\
        & - a_{13}a_{22}a_{31} - a_{11}a_{23}a_{32} - a_{12}a_{21}a_{33}
    \end{aligned}
\end{equation}


Para el caso de una matriz de orden 4, podemos intentar copiando las primeras tres columnas de la matriz y seguir la misma regla de la suma y resta de diagonales

\begin{equation}
    \begin{aligned}
        \left(
        \begin{matrix}
            a_{11} & a_{12} & a_{13} & a_{14} \\
            a_{21} & a_{22} & a_{23} & a_{24} \\
            a_{31} & a_{32} & a_{33} & a_{34} \\
            a_{41} & a_{42} & a_{43} & a_{44} \\
        \end{matrix}
        \middle|
        \begin{matrix}
            a_{11} & a_{12} & a_{13} \\
            a_{21} & a_{22} & a_{23} \\
            a_{31} & a_{32} & a_{33} \\
            a_{41} & a_{42} & a_{43} \\
        \end{matrix} 
        \right)
        & = a_{11}a_{22}a_{33}a_{44} + a_{12}a_{23}a_{34}a_{41} \\
        & + a_{13}a_{24}a_{31}a_{42} + a_{14}a_{21}a_{32}a_{43} \\
        & - a_{14}a_{23}a_{32}a_{41} - a_{11}a_{24}a_{33}a_{42} \\
        & - a_{12}a_{21}a_{34}a_{43} - a_{13}a_{22}a_{31}a_{44}
    \end{aligned}
\end{equation}

Ahora, comprobaremos usando el método de cofactores

\begin{equation}
    \begin{aligned}
        \left|
        \begin{matrix}
            a_{11} & a_{12} & a_{13} & a_{14} \\
            a_{21} & a_{22} & a_{23} & a_{24} \\
            a_{31} & a_{32} & a_{33} & a_{34} \\
            a_{41} & a_{42} & a_{43} & a_{44} \\
        \end{matrix}
        \right|
        & = a_{11}
            \left|
            \begin{matrix}
                a_{22} & a_{23} & a_{24} \\
                a_{32} & a_{33} & a_{34} \\
                a_{42} & a_{43} & a_{44} \\
            \end{matrix}
            \right|
        - a_{12}
            \left|
            \begin{matrix}
                a_{21} & a_{23} & a_{24} \\
                a_{31} & a_{33} & a_{34} \\
                a_{41} & a_{43} & a_{44} \\
            \end{matrix}
            \right|
        \\    
        & + a_{13}
            \left|
            \begin{matrix}
                a_{21} & a_{22} & a_{24} \\
                a_{31} & a_{32} & a_{34} \\
                a_{41} & a_{42} & a_{44} \\
            \end{matrix}
            \right|
        - a_{14}
            \left|
            \begin{matrix}
                a_{21} & a_{22} & a_{23} \\
                a_{31} & a_{32} & a_{33} \\
                a_{41} & a_{42} & a_{43} \\
            \end{matrix}
            \right|
        \\
        & = a_{11}
            \left(
                a_{22}
                \left|
                \begin{matrix}
                    a_{33} & a_{34} \\
                    a_{43} & a_{44} \\
                \end{matrix}
                \right|
            \right)
        - a_{12}
            \left|
            \begin{matrix}
                a_{21} & a_{23} & a_{24} \\
                a_{31} & a_{33} & a_{34} \\
                a_{41} & a_{43} & a_{44} \\
            \end{matrix}
            \right|
        \\    
        & + a_{13}
            \left|
            \begin{matrix}
                a_{21} & a_{22} & a_{24} \\
                a_{31} & a_{32} & a_{34} \\
                a_{41} & a_{42} & a_{44} \\
            \end{matrix}
            \right|
        - a_{14}
            \left|
            \begin{matrix}
                a_{21} & a_{22} & a_{23} \\
                a_{31} & a_{32} & a_{33} \\
                a_{41} & a_{42} & a_{43} \\
            \end{matrix}
            \right|
        \\
    \end{aligned}
\end{equation}

Podemos observar que estos no son iguales, por lo que la regla de Sarrus no siempre es válida para determinantes de orden distinto a 3, aunque podemos pensar en algún caso especial; supongamos que a = ..., etc., entonces: 


En resumen, el método de cofactores es aplicable para obtener el determinante de cualquier matriz cuadrada, a diferencia de la regla de Sarrus que en realidad sólo es un caso especial de la regla de Leibniz y no siempre es válida para matrices de orden distinto a tres.

Respondiendo a las preguntas del laboratorio:
1. No es posible aplicar el método de la lluvia para cualquier matriz 4x4, aunque puede ser aplicable si satisface que:

2. No es posible ya que 
\end{document}