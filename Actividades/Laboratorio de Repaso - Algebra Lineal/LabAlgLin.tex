\documentclass[12pt, letterpaper]{article}
\usepackage[utf8]{inputenc}
\usepackage[letterpaper]{geometry}
\geometry{top=1.5cm, bottom=2.5cm, left=2.5cm, right=2.5cm}
\usepackage{amsmath}
\usepackage{fancyhdr}
\usepackage{graphicx}
\pagestyle{fancy}
\fancypagestyle{plain}{}
\fancyhf{}
\lfoot[]{Inteligencia Artificial}
\rfoot[]{Enero - Junio 2025}
\renewcommand{\headrulewidth}{0pt}
\renewcommand{\footrulewidth}{1pt}
\usepackage{amsmath,amssymb}
\title{Laboratorio de Álgebra Lineal}
\author{Aldo Hernández}
\date{Febrero 2025}

\begin{document}
\maketitle

\section{Operaciones con matrices y determinantes}
\begin{enumerate}
    \item Consideremos la siguiente matriz F
    \begin{equation*}
        \begin{aligned}
            F
            & =
            \begin{pmatrix}
                1 & 2 & 3 &\bigm| & 1 & 0 & 0 \\
                0 & 1 & 4 &\bigm| & 0 & 1 & 0 \\
                5 & 6 & 0 &\bigm| & 0 & 0 & 1 
            \end{pmatrix}
            \sim
            \begin{pmatrix}
                1 & 2 & 3 &\bigm| & 1 & 0 & 0 \\
                5 & 6 & 0 &\bigm| & 0 & 0 & 1 \\
                0 & 1 & 4 &\bigm| & 0 & 1 & 0 
            \end{pmatrix} \\
            & \sim
            \begin{pmatrix}
                1 & 2 & 3 &\bigm| & 1 & 0 & 0 \\
                0 & -4 & -15 &\bigm| & -5 & 0 & 1 \\
                0 & 1 & 4 &\bigm| & 0 & 1 & 0 
            \end{pmatrix} 
            \sim
            \begin{pmatrix}
                1 & 2 & 3 &\bigm| & 1 & 0 & 0 \\
                0 & 1 & \frac{15}{4} &\bigm| & \frac{5}{4} & 0 & \frac{-1}{4} \\
                0 & 1 & 4 &\bigm| & 0 & 1 & 0 
            \end{pmatrix} \\
            & \sim
            \begin{pmatrix}
                1 & 0 & \frac{-9}{2} &\bigm| & \frac{-3}{2} & 0 & \frac{1}{2} \\
                0 & 1 & \frac{15}{4} &\bigm| & \frac{5}{4} & 0 & \frac{-1}{4} \\
                0 & 0 & \frac{1}{4} &\bigm| & \frac{-5}{4} & 1 & \frac{1}{4}
            \end{pmatrix}
            \sim
            \begin{pmatrix}
                1 & 0 & \frac{-9}{2} &\bigm| & \frac{-3}{2} & 0 & \frac{1}{2} \\
                0 & 1 & \frac{15}{4} &\bigm| & \frac{5}{4} & 0 & \frac{-1}{4} \\
                0 & 0 & 1 &\bigm| & -5 & 4 & 1
            \end{pmatrix} \\
            & \sim
            \begin{pmatrix}
                1 & 0 & 0 &\bigm| & -24 & 18 & 5 \\
                0 & 1 & 0 &\bigm| & 20 & -15 & -4 \\
                0 & 0 & 1 &\bigm| & -5 & 4 & 1
            \end{pmatrix} \\
            \Rightarrow F^{-1}
            & = 
            \begin{pmatrix}
                -24 & 18 & 5 \\
                20 & -15 & -4 \\
                -5 & 4 & 1
            \end{pmatrix}
        \end{aligned}
    \end{equation*}
    % \item Sea $A$ una matriz cuadrada cualquiera de orden $n$, sea B otra matriz cuadrada no necesariamente igual que $A$ de orden $n$ y sea $C = AB$, tenemos que
    % \begin{equation}
    %     \det(A) = 
    %     \begin{vmatrix}
    %         a_{11} & \cdots & a_{1n} \\
    %         \vdots & \ddots & \vdots \\
    %         a_{n1} & \cdots & a_{nn}
    %     \end{vmatrix} =
    %     \sum_{j=1}^{n} (-1)^{1+j}a_{1,j}A_{1,j}
    % \end{equation}
    % \begin{equation}
    %     \det(B) = 
    %     \begin{vmatrix}
    %         b_{11} & \cdots & b_{1n} \\
    %         \vdots & \ddots & \vdots \\
    %         b_{n1} & \cdots & b_{nn}
    %     \end{vmatrix} =
    %     \sum_{k=1}^{n} (-1)^{1+k}b_{1,k}B_{1,k}
    % \end{equation}
    % \begin{equation}
    %     \det(C) = 
    %     \begin{vmatrix}
    %         a_{11}b_{11} + \cdots + a_{1n}b_{n1} & \cdots & a_{11}b_{1n} + \cdots + a_{1n}b_{nn} \\
    %         \vdots & \ddots & \vdots \\
    %         a_{n1}b_{11} + \cdots + a_{nn}b_{n1} & \cdots & a_{n1}b_{1n} + \cdots + a_{nn}b_{nn}
    %     \end{vmatrix} =
    %     \sum_{m=1}^{n} (-1)^{1+m}c_{1,m}C_{1,m}
    % \end{equation}

    % \newpage

    % Entonces si multiplicamos el determinante de $A$ y $B$ (ecuaciones 1 y 2) obtenemos lo siguiente
    
    % \begin{equation*}
    %     \begin{aligned}
    %         \sum_{j=1}^{n} (-1)^{1+j}a_{1,j}A_{1,j}
    %         \sum_{k=1}^{n} (-1)^{1+k}b_{1,k}B_{1,k}
    %         & = 
    %         \sum_{j=1}^{n} \sum_{k=1}^{n} a_{1,j}b_{1,k}A_{1,j}B_{1,k}
    %     \end{aligned}
    % \end{equation*}

\end{enumerate}

\newpage

\section{Sistemas de ecuaciones lineales}
\begin{enumerate}
    \item Pendiente (Gauss-Seidel?? zzz)
    \item Para el sistema homogéneo, podemos definir las siguientes matrices
    \begin{equation*}
        A =
        \begin{pmatrix}
            1 & 2 & 3 \\
            2 & 4 & 6 \\
            3 & 6 & 9
        \end{pmatrix},
        X =
        \begin{pmatrix}
            x \\
            y \\
            z
        \end{pmatrix},
        B =
        \begin{pmatrix}
            0 \\
            0 \\
            0
        \end{pmatrix}
    \end{equation*}
    De forma que $CX = B$, pero observamos que todos los renglones de C son múltiplos, es decir, son linealmente dependientes, por lo que al realizar operaciones elementales con $A$, eventualmente llegaremos al siguiente resultado
    \begin{equation*}
        \begin{aligned}
            A
            & \sim
            \begin{pmatrix}
                1 & 2 & 3 \\
                0 & 0 & 0 \\
                0 & 0 & 0
            \end{pmatrix} \\
            \Rightarrow x
            & = -2y-3z \\
            \Rightarrow X
            & =
            \begin{pmatrix}
                -2y-3z \\
                y \\
                z
            \end{pmatrix} \\
            & = y
            \begin{pmatrix}
                -2 \\
                1 \\
                0
            \end{pmatrix}
            + z
            \begin{pmatrix}
                -3 \\
                0 \\
                1
            \end{pmatrix}
        \end{aligned}
    \end{equation*}
    Entonces decimos que los vectores constantes
    $\begin{pmatrix}
        -2 \\
        1 \\
        0
    \end{pmatrix}$ y
    $\begin{pmatrix}
        -3 \\
        0 \\
        1
    \end{pmatrix}$
    son soluciones básicas del sistema de ecuaciones y que el sistema es compatible indeterminado con dos variables libres. Es por esto que hay soluciones infinitas dadas por el siguiente conjunto solución
    \begin{equation*}
        \left\{ 
            x = -2y-3z, y, z | x,y,z \in \mathbb{R} 
        \right\}
    \end{equation*}
\end{enumerate}

\newpage

\section{Espacios vectoriales y auto-valores/auto-vectores}
\begin{enumerate}
    \item Vemos que los vectores del subespacio dado son linealmente dependiemntes
    \begin{equation*}
        \left\{
            \begin{pmatrix}
                1 \\
                2 \\
                3
            \end{pmatrix},
            \begin{pmatrix}
                2 \\
                4 \\
                6
            \end{pmatrix},
            \begin{pmatrix}
                3 \\
                6 \\
                9
            \end{pmatrix}
        \right\}
        =
        \left\{
            \begin{pmatrix}
                1 \\
                2 \\
                3
            \end{pmatrix},
            2
            \begin{pmatrix}
                1 \\
                2 \\
                3
            \end{pmatrix},
            3
            \begin{pmatrix}
                1 \\
                2 \\
                3
            \end{pmatrix}
        \right\}
    \end{equation*}
    Entonces podemos concluir que una base para el subespacio vectorial es
    \begin{equation*}
        B =
        \left\{
            \begin{pmatrix}
                1 \\
                2 \\
                3
            \end{pmatrix}
        \right\}, \dim B = 1
    \end{equation*}
    \item Para hallar los eigenvalores de $G$ tenemos que igualar el siguiente determinante a cero
    \begin{equation*}
        \begin{aligned}
            |G - \lambda I|
            & = \begin{vmatrix}
                5-\lambda & -2 \\
                -2 & 5-\lambda
            \end{vmatrix} \\
            & = (5-\lambda)^2 - (-2)^2 \\
            \Rightarrow 0 & = (5-\lambda)^2 - (-2)^2 \\
            4 & = (5-\lambda)^2 \\
            \therefore \lambda_{1} & = 3 \\
            \therefore \lambda_{2} & = 7
        \end{aligned}
    \end{equation*}
    Para encontrar los eigenvectores, basta con sustituir los eigenvalores en $G-\lambda I$
    \begin{equation*}
        \begin{aligned}
            G - 3I =
            \begin{pmatrix}
                2 & -2 \\
                -2 & 2
            \end{pmatrix}
            & \sim
            \begin{pmatrix}
                2 & -2 \\
                0 & 0
            \end{pmatrix} \\
            \Rightarrow x_{1} & = x_{2}
        \end{aligned}
    \end{equation*}
    \begin{equation*}
        \begin{aligned}
            G - 7I =
            \begin{pmatrix}
                -2 & -2 \\
                -2 & -2
            \end{pmatrix}
            & \sim
            \begin{pmatrix}
                -2 & -2 \\
                0 & 0
            \end{pmatrix} \\
            \Rightarrow x_{1} & = -x_{2}
        \end{aligned}
    \end{equation*}
    \begin{equation*}
        \therefore 
        \begin{pmatrix}
            1 \\
            1
        \end{pmatrix}_{\lambda = 3}
        \land 
        \begin{pmatrix}
            -1 \\
            1
        \end{pmatrix}_{\lambda = 7},
    \end{equation*}
\end{enumerate}

\newpage

\section{Aplicaciones en IA: reducción de dimensionalidad}
\begin{enumerate}
    \item PCA
    \item SVD
    \item DL w NN
    \item VS in AI
\end{enumerate}

\end{document}